%================================================================
% H1 paper
%================================================================
\RequirePackage{lineno}
\documentclass[12pt]{article}


\usepackage{epsfig}
\usepackage{hhline}
\usepackage{amsmath}
\usepackage{amssymb}
\usepackage{color}
\usepackage{xspace}
%\usepackage{xfrac}
\usepackage{paralist} % compactitem
\usepackage{caption}
%\captionsetup[figure]{font=footnotesize,labelfont=footnotesize}
\captionsetup{font=small,labelfont=small}

\usepackage{bm} % 'bold' math symbols (better use \usepackage{newtxtext,newtxmath}, if available on latex distribution)
\usepackage{acronym}

%%%%%%%%%%%%% Comment the next two lines to remove the line numbering
\usepackage[]{lineno}
\linenumbers
%%%%%%%%%%%%%%

\usepackage{hyperref} % has to be last package loaded
\hypersetup{colorlinks=true, urlcolor=blue}
\usepackage{cite} % DB: enable also clickable references (must be loaded after hyperref)
\hypersetup{
%  colorlinks,
%  citecolor=blue,
%  linkcolor=red,
%  urlcolor=blue
  citecolor=[rgb]{0.,0.,0.5},
  linkcolor=[rgb]{0.6,0.,0.0},
  urlcolor=[rgb]{0.,0.,0.5}
  }


%%%%%%%%%%%%%% H1 preliminary
%\renewcommand{\topfraction}{1.0}
%\renewcommand{\bottomfraction}{1.0}
%\renewcommand{\textfraction}{0.0}
%\renewcommand{\arraystretch}{1.3} % make lines a bit larger for tables 
%%%%%%%%%%%%%%

%%%%%%%%%%%%%% H1 paper layout %%%%%%%%%%%%%%
\renewcommand{\topfraction}{1.0}
\renewcommand{\bottomfraction}{1.0}
\renewcommand{\textfraction}{0.0}
\renewcommand{\arraystretch}{1.25} % make lines a bit larger for tables
%\newlength{\dinwidth}
%\newlength{\dinmargin}
%\setlength{\dinwidth}{21.0cm}
%\textheight23.5cm \textwidth16.0cm
%\setlength{\dinmargin}{\dinwidth}
%\setlength{\unitlength}{1mm}
%\addtolength{\dinmargin}{-\textwidth}
%\setlength{\dinmargin}{0.5\dinmargin}
%\oddsidemargin -1.0in
%\addtolength{\oddsidemargin}{\dinmargin}
%\setlength{\evensidemargin}{\oddsidemargin}
%\setlength{\marginparwidth}{0.9\dinmargin}
\marginparsep 8pt \marginparpush 5pt
\topmargin -42pt
\headheight 12pt
\headsep 30pt \footskip 24pt
\parskip 3mm plus 2mm minus 2mm

% do not indent first line of paragraph!
%\setlength{\parindent}{0pt}
\usepackage{parskip}
%%%%%%%%%%%%%%%%%%%%%%%%%%%%%%%%%%%%%%%%%%%%%




%%% contains utf-8, see: http://inspirehep.net/info/faq/general#utf8
%%% add \usepackage[utf8]{inputenc} to your latex preamble
\usepackage[utf8]{inputenc}
%\bibliographystyle{plain}%Choose a bibliograhpic style          
\bibliographystyle{utphys}%Choose a bibliograhpic style          

\newlength{\dinwidth}
\newlength{\dinmargin}
\setlength{\dinwidth}{21.0cm}
\textheight23.5cm \textwidth16.0cm
\setlength{\dinmargin}{\dinwidth}
\setlength{\unitlength}{1mm}
\addtolength{\dinmargin}{-\textwidth}
\setlength{\dinmargin}{0.5\dinmargin}
\oddsidemargin -1.0in
\addtolength{\oddsidemargin}{\dinmargin}
\setlength{\evensidemargin}{\oddsidemargin}
\setlength{\marginparwidth}{0.9\dinmargin}
\marginparsep 8pt \marginparpush 5pt
\topmargin -42pt
\headheight 12pt
\headsep 30pt \footskip 24pt
\parskip 3mm plus 2mm minus 2mm
\setlength{\parindent}{0.0cm} 
\newcommand{\picob}{\mbox{{\rm ~pb}}}
\newcommand{\QQ}  {\mbox{${Q^2}$}}

\newcommand{\NNLOJET}{NNLO\protect\scalebox{0.8}{JET}\xspace}

%===============================title page=============================

% Some useful tex commands
%
%\def\GeV{\hbox{$\;\hbox{\rm GeV}$}}
%\def\MeV{\hbox{$\;\hbox{\rm MeV}$}}
%\def\TeV{\hbox{$\;\hbox{\rm TeV}$}}

\newcommand{\pb}{\rm pb}
\newcommand{\cm}{\rm cm}
\newcommand{\hdick}{\noalign{\hrule height1.4pt}}

\begin{document}
\pagestyle{empty}

\newcommand{\GeVsq}{\ensuremath{\mathrm{GeV}^2} }
\newcommand{\GeV}{\ensuremath{\mathrm{GeV}} }
\newcommand{\pt}{\ensuremath{P_{T}}}
\newcommand{\PP}{\ensuremath{\mathcal{P}}}
%\newcommand{\ptAvg}{\ensuremath{\langle P_T^{jet}\rangle}}
\newcommand{\Qsq}{\ensuremath{Q^{2}}}
%% unfolding
\newcommand{\chisq}{\ensuremath{\chi^{2}}}
\newcommand{\chisqA}{\ensuremath{\chi_{\rm A}^{2}}}
\newcommand{\chisqL}{\ensuremath{\chi_{\rm L}^{2}}}
\newcommand{\ndf}{\ensuremath{n_{\rm dof}}}
\newcommand{\A}{\ensuremath{\bm{A}}}
\newcommand{\V}{\ensuremath{\bm{V}}}
\newcommand{\B}{\ensuremath{\bm{B}}}
\newcommand{\J}{\ensuremath{\bm{J}}}
\newcommand{\N}{\ensuremath{\bm{N}}}
\newcommand{\LL}{\ensuremath{\bm{L}}}

\newcommand{\etajet}{\ensuremath{\eta_{\rm lab}^{\rm jet}}}
\newcommand{\ptjet}{\ensuremath{P_{\rm T}^{\rm jet}}}
\newcommand{\meanpt}{\ensuremath{\langle P_{\rm T} \rangle}}
\newcommand{\etalab}{\ensuremath{\eta_{\rm lab}^{\rm jet}}}
\newcommand{\Mjj}{\ensuremath{m_{12}}}
\newcommand{\meanptdi}{\ensuremath{\langle P_{\mathrm{T}} \rangle_{2}}\xspace}
\newcommand{\meanpttri}{\ensuremath{\langle P_{\mathrm{T}} \rangle_{3}}\xspace}
\newcommand{\mz}{\ensuremath{m_{\rm Z}}\xspace}
\newcommand{\as}{\ensuremath{\alpha_{\rm s}}\xspace}
\newcommand{\asmz}{\ensuremath{\as(\mz)}\xspace}
\newcommand{\asmzPDF}{\ensuremath{\as^{\rm PDF}(\mz)}\xspace}
\newcommand{\asmzf}{\ensuremath{\as^{\Gamma}(\mz)}\xspace}
\newcommand{\tilmu}{\ensuremath{\tilde{\mu}}\xspace}
\newcommand{\etal}{{\it{et al.}}}
\newcommand{\mur}{\ensuremath{\mu_{\rm R}}\xspace}
\newcommand{\muf}{\ensuremath{\mu_{\rm F}}\xspace}
\newcommand{\mup}{\ensuremath{\mu_{0}}\xspace}
\newcommand{\murf}{\ensuremath{\mu_{\rm R/F}}\xspace}
\newcommand{\asmur}{\ensuremath{\alpha_{\rm s}(\mur)}\xspace}
\newcommand{\chad}{\ensuremath{c_{\rm had}}\xspace}
\newcommand{\ord}{\ensuremath{\mathcal{O}}\xspace}
\newcommand{\PDFasResult}{0.1142}

\renewcommand{\contentsname}{Content \footnotesize (only for editorial purposes)}


% % % Journal macro
% % \def\Journal#1#2#3#4{{#1}~{\bf #2} (#3) #4}
% % %\def\NCA{\em Nuovo Cimento}
% % %\def\NIM{\em Nucl. Instrum. Methods}
% % %\def\NIMA{{\em Nucl. Instrum. Methods} {\bf A}}
% % %\def\NPB{{\em Nucl. Phys.}   {\bf B}}
% % %\def\PLB{{\em Phys. Lett.}   {\bf B}}
% % %\def\PRL{\em Phys. Rev. Lett.}
% % %\def\PRD{{\em Phys. Rev.}    {\bf D}}
% % %\def\ZPC{{\em Z. Phys.}      {\bf C}}
% % %\def\EJC{{\em Eur. Phys. J.} {\bf C}}
% % %\def\CPC{\em Comp. Phys. Commun.}
% % %
% % \def\NPB{Nucl. Phys.~}
% % \def\PRL{Phys. Rev. Lett.~}
% % \def\EPJC{Eur. Phys. J.~}
% % \def\PLB{Phys. Lett.~}
% % \def\NIM{Nucl. Instrum. Meth.~}
% % \def\PRD{Phys. Rev.~}
% % \def\JHEP{JHEP~}
% % \def\PROC{Conf. Proc.~}
% % \def\CPC{Comp. Phys. Commun.~}
% % 




%%%%%%%%%%%%%%%%%%%%%%%%%%%%%%%%%%%%%%% title page %%%%%%%%%%%%%%%%%%%%%%%%%%%%%%%%%%%%%%%%
\begin{titlepage}
\noindent
\begin{flushleft}
{\tt H1prelim-19-011 DRAFT ! \hfill    ISSN abcd-abcd} \\
{\tt March 2019}                  \\
\end{flushleft}

\noindent
Date:   ~   \ \ March 2019 %31.\,08.\,2017      \\
Version:~   0.0 \\
Editors:~   D.~Britzger, R.~\v{Z}leb\v{c}\'{i}k (daniel.britzger@desy.de, zlebcr@mail.desy.de) \\
Referees:  A.~B, B.~C \\
%Final reading scheduled for ....         \\
\noindent

\vspace{1cm}
\begin{center}
\begin{Large}

{\bf
   A NNLO Combined fit of the inclusive and dijet data

}

\vspace{1.5cm}

H1 Collaboration%\footnote{}

\end{Large}
\end{center}

\vspace{1.5cm}


\begin{abstract}
\noindent
%Abstract
A new combined fit of diffractive parton distribution functions (DPDFs) to the HERA inclusive and jet data in diffractive deep-inelastic scattering (DDIS) at next-to-next-to-leading order accuracy (NNLO) is presented.
The inclusion of the most comprehensive dijet cross section data, together with their NNLO predictions, provide enhanced constraints to the gluon component of the DPDF, which is of particular importance for diffractive PDFs.
Compared to the previous HERA fits, the presented fit includes the high-precision HERA-II data, which represents 40times higher luminosity for inclusive sample and 6times higher luminosity for the jet sample.
In addition to the inclusive DDIS sample at the nominal centre-of-mass energy $\sqrt{s}=319$, also the inclusive data at 252, 225 GeV are included into the fit.
The extracted DPDFs are compared to the alternative existing DPDFs, and are used to predict cross sections for large number of the available jet measurements and different observables.
\noindent
\end{abstract}


\begin{center} Submitted to DIS19 conference, Torino \end{center}

\end{titlepage}
%% ------------------------------------------------------------------------------ %%





%% ------------------------------------------------------------------------------ %%
\clearpage
%\tableofcontents
%\clearpage
%% ------------------------------------------------------------------------------ %%

\pagestyle{plain} %% page numbers


%%%%%%%%%%%%%%%%%%%%%%%%%%%%%%%%%%%%%%%%%%%%%%%%%%%%%%%%%%%%
%                    Introduction
%%%%%%%%%%%%%%%%%%%%%%%%%%%%%%%%%%%%%%%%%%%%%%%%%%%%%%%%%%%%
\section{Introduction}
Intro



%%%%%%%%%%%%%%%%%%%%%%%%%%%%%%%%%%%%%%%%%%%%%%%%%%%%%%%%%%%%
%                    Data
%%%%%%%%%%%%%%%%%%%%%%%%%%%%%%%%%%%%%%%%%%%%%%%%%%%%%%%%%%%%
%\clearpage
\section{Data selection}
For the present analysis measurements of jet cross sections and inclusive DIS cross sections in
lepton-proton collisions performed by the H1 experiment at HERA are exploited.

\paragraph{Inclusive diffractive DIS data}
incl. DDIS data

\paragraph{Dijet data in dffractive DIS}
Dijet data

\begin{table}[tbhp]
  \footnotesize
  %\scriptsize
  \begin{center}
    \begin{tabular}{cccccc}
%      \multicolumn{6}{c}{\bf Kinematic range of H1 jet data} \\
      \hline
      \multicolumn{1}{c}{Data set} & $\sqrt{s}$ & int. $\mathcal{L}$ & DIS kinematic &  Inclusive jets &  Dijets   \\  
      \multicolumn{1}{c}{[ref.]}  & $[\GeV]$   & $[{\rm pb}^{-1}]$  &  range        &                 &   $n_{\rm jets}\ge2 $  \\   
      \hline
      $300\,\GeV$ & 300 & 33& $150<\Qsq<5000\,\GeVsq$  & $7<\ptjet<50\,\GeV$ & $\ptjet>7\,\GeV$  \\
      \cite{Adloff:2000tq} &          & & $0.2<y<0.6$             &            & $8.5<\meanpt<35\,\GeV$    \\
      \hline
      HERA-I    & 319  &43.5 & $5<\Qsq<100\,\GeVsq$   &   $5<\ptjet<80\,\GeV$ & $5<\ptjet<50\,\GeV$  \\
\cite{Aaron:2010ac}  &      &     & $0.2<y<0.7$                &                      & $5<\meanpt<80\,\GeV$  \\
                &           &     &                            &                      & $\Mjj>18\,\GeV$  \\
                &           &     &                            &                      & $(\meanpt>7\,\GeV)^*$ \\
      \hline
      HERA-I    & 319  &65.4 & $150<\Qsq<15000\,\GeVsq$   &   $5<\ptjet<50\,\GeV$ & $-$  \\
\cite{Aktas:2007aa} &  &     & $0.2<y<0.7$             &                      &  \\
      \hline
      HERA-II   & 319  & 290& $5.5<\Qsq<80\,\GeVsq$        & $4.5<\ptjet<50\,\GeV$ & $\ptjet>4\,\GeV$  \\
\cite{Andreev:2016tgi}&& & $0.2<y<0.6$                &                      & $5<\meanpt<50\,\GeV$  \\
      \hline
      HERA-II   & 319  & 351& $150<\Qsq<15000\,\GeVsq$     &   $5<\ptjet<50\,\GeV$ & $5<\ptjet<50\,\GeV$  \\
 \cite{Andreev:2014wwa,Andreev:2016tgi}               &           & & $0.2<y<0.7$                &                      & $7<\meanpt<50\,\GeV$  \\
                &           & &                            &                      & $\Mjj>16\,\GeV$  \\
      \hline
    \end{tabular}
    \caption{
      An old table to be used as template.
    }
    \label{tab:datasets}
    \end{center}
\end{table}


%%%%%%%%%%%%%%%%%%%%%%%%%%%%%%%%%%%%%%%%%%%%%%%%%%%%%%%%%%%%
%                    Theory
%%%%%%%%%%%%%%%%%%%%%%%%%%%%%%%%%%%%%%%%%%%%%%%%%%%%%%%%%%%%
%\clearpage
\begin{boldmath}
\section{Theory}
\end{boldmath}
\label{sec:theory}


%%%%%%%%%%%%%%%%%%%%%%%%%%%%%%%%%%%%%%%%%%%%%%%%%%%%%%%%%%%%
%                    Methodology
%%%%%%%%%%%%%%%%%%%%%%%%%%%%%%%%%%%%%%%%%%%%%%%%%%%%%%%%%%%%
\section{Methodology}
\label{sec:method}


%%%%%%%%%%%%%%%%%%%%%%%%%%%%%%%%%%%%%%%%%%%%%%%%%%%%%%%%%%%%
%                      Results
%%%%%%%%%%%%%%%%%%%%%%%%%%%%%%%%%%%%%%%%%%%%%%%%%%%%%%%%%%%%
%\clearpage
\subsection{Results: H1DPDF2019}
%%%%%%%%%%%%%%%%%%%%%%%%%%%%%%%%%%%%%%%%%%%%%%%%%%%%%%%%%%%%
%   main results
\paragraph{Fit to inclusive DDIS data}
\label{sec:resultsIncl}


%%%%%%%% Fig.:   %%%%%%%%%%%%%%%%%%%%
\begin{figure}[ht]
\begin{center}
%  \includegraphics[width=0.8\textwidth]{plots/ratio.pdf}
\end{center}
\caption{
  test
}
\label{fig:DPDFratio}
\end{figure}






\paragraph{Fit to inclusive DDIS and dijet data}
\label{sec:resultsJets}


%%%%%%%%%%%%%%%%%%%%%%%%%%%%%%%%%%%%%%%%%%%%%%%%%%%%%%%%%%%%
%                    Summary
%%%%%%%%%%%%%%%%%%%%%%%%%%%%%%%%%%%%%%%%%%%%%%%%%%%%%%%%%%%%
%\clearpage
\section{Summary}
The new next-to-next-to-leading order pQCD calculations (NNLO) for dijet
production cross sections in diffractive neutral current DIS are exploited for a
determination of diffractive parton distribution functions (DPDFs) inclusive
dijet and inclusive NC DDIS cross section measurements published by the H1 Collaboration. 







%%%%%%%%%%%%%%%%%%%%%%%%%%%%%%%%%%%%%%%%%%%%%%%%%%%%%%%%%%%%%%%%%%%%%
%         Acknowledgements
%%%%%%%%%%%%%%%%%%%%%%%%%%%%%%%%%%%%%%%%%%%%%%%%%%%%%%%%%%%%%%%%%%%%%
%\clearpage
%\section*{Acknowledgements}
%We are grateful to the HERA machine group whose outstanding
%efforts have made this experiment possible.
%We thank the engineers and technicians for their work in constructing
%and maintaining the H1 detector, our funding agencies for
%financial support, the DESY technical staff for continual assistance
%and the DESY directorate for support and for the
%hospitality which they extend to the non--DESY
%members of the collaboration.
%
%We would like to give credit to all partners contributing to the EGI
%computing infrastructure for their support for the H1 Collaboration. 
%
%We express our thanks to all those involved in securing not only the
%H1 data but also the software and working environment for long term
%use allowing the unique H1 data set to continue to be explored in the
%coming years. The transfer from experiment specific to central
%resources with long term support, including both storage and batch
%systems has also been crucial to this enterprise. We therefore also
%acknowledge the role played by DESY-IT and all people involved during
%this transition and their future role in the years to come. 
%
%{\color{red} Mark Sutton: Here you will need to add an acknowledgement
%  to the IPPP for the Associateship.}



%%%%%%%%%%%%%%%%%%%%%%%%%%%%%%%%%%%%%%%%%%%%%%%%%%%%%%%%%%%%
%                    bib
%%%%%%%%%%%%%%%%%%%%%%%%%%%%%%%%%%%%%%%%%%%%%%%%%%%%%%%%%%%%
\clearpage
\bibliography{H1prelim-19-011}




%%%%%%%%%%%%%%%%%%%%%%%%%%%%%%%%%%%%%%%%%%%%%%%%%%%%%%%%%%%%
%                    tables
%%%%%%%%%%%%%%%%%%%%%%%%%%%%%%%%%%%%%%%%%%%%%%%%%%%%%%%%%%%%
\clearpage
%\section*{Table of results}



%%%%%%%%%%%%%%%%%%%%%%%%%%%%%%%%%%%%%%%%%%%%%%%%%%%%%%%%%%%%
%                    figures
%%%%%%%%%%%%%%%%%%%%%%%%%%%%%%%%%%%%%%%%%%%%%%%%%%%%%%%%%%%%
\clearpage
%\section*{Figures}

\end{document}
